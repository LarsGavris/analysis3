\chapter{Konvergenzsätze und $L^n$-Räume}
  \begin{example}
    Punktweise Konvergenz reicht nicht für Konvergenz der Integrale.\\
    Für $\epsilon > 0$ sei $f_{\epsilon}: \mathbb{R} \to \mathbb{R}, f_{\epsilon} = \dfrac{1}{2\epsilon} \chi_{[-\epsilon, \epsilon]}$\\
    Es gilt $f_{\epsilon}(x) = 0$ für $\epsilon < |x|$\\
    $\implies f(x) := \lim\limits_{\epsilon \downarrow 0} f_{\epsilon}(x) = \begin{cases}
      0 & \text{, für } x \neq 0\\
      \infty & \text{, für } x = 0
    \end{cases}$\\
    Weiter $\int f_{\epsilon} d\lambda^1 = \dfrac{1}{2\epsilon} \lambda^1([-\epsilon, \epsilon]) = 1 \ \forall \epsilon > 0$\\
    $\implies \int f d\lambda^1 = 0 < 1 = \lim\limits_{\epsilon \downarrow 0} f_{\epsilon} d\lambda^1$
  \end{example}

  \begin{theorem}[Lemma von Fatou]
    $f_k: X \to [0,\infty]$ Folge von $\mu$-messbaren Funktionen.\\
    Für $f: X \to \bar{\mathbb{R}}, f(x) = \liminf\limits_{k \to \infty} f_k(x)$ gilt:
    \begin{align*}
      \int f d\mu \leq \liminf\limits_{k \to \infty} \int f_k d\mu
    \end{align*}
  \end{theorem}

  \begin{proof}
    siehe Aufschrieb
  \end{proof}

  \begin{theorem}[Dominierte Konvergenz bzw. Satz von Lebesgue]
    $f_1, f_2, ...$ Folge von $\mu$-messbare Funktionen und $f(x) = \lim\limits_{k \to \infty} f_k(x)$ für $\mu$-fast alle $x \in X$. Es gebe eine integrierbare Funktion $g: X \to [0, \infty]$ mit $\sup\limits_{k \in \mathbb{N}} |f_k(x)| \leq g(x)$ für $\mu$-fast alle $x$. Fann ist $f$ integrierbar und $\int f d\mu = \lim\limits_{k \to \infty} \int f_k d\mu$.\\
    Es gilt sogar $||f_k \cdot f||_{L^1(y)} := \int |f_k -f| d\mu \to 0$
  \end{theorem}

  \begin{proof}
    siehe Aufschrieb
  \end{proof}

  \begin{remark}[Anwendung]
    Vergleich Riemann-$\int$ mit Lebesgue-$\int$\\
    Sei $I=[a,b]$ kompaktes Intervall, $f:I \to \mathbb{R}$ beschränkt. Unterteilungspunkte $a = x_0 \leq ... \leq x_N = b$ $\to$ Zerlegung $Z$ von $I$ mit Teilintervallen $I_j = [x_{j-1}, x_j]$\\
    $\bar{S}_Z(f) = \sum\limits_{j=1}^N (\sup\limits_{I_j} f) (x_j - x_{j-1}), \ \ \ \underbar{S}_Z(f)= \sum\limits_{j=1}^N (\inf\limits_{I_j} f)(x_j-x_{j-1})$\\
    Für Zerlegungen $Z_1, Z_2$ mit Verfeinerung $Z_1 \cup Z_2$\\
    $\implies \underbar{S}_{Z_1}(f) \leq \underbar{S}_{Z_1 \cup Z_2}(f) \leq \bar{S}_{Z_1 \cup Z_2}(f) \leq \bar{S}_{Z_2}(f)$\\
    $f$ heißt \textbf{Riemann-integrierbar} mit Integral $\int\limits_a^b f(x) dx = S$, falls gilt:\\
    $\sup\limits_Z \underbar{S}_Z(f) = \inf\limits_Z \bar{S}_Z(f) = S$
  \end{remark}

  \begin{theorem}
    $f: I \to \mathbb{R}$ beschränkt auf kompaktem Intervall $I=[a,b]$. Dann gilt:\\
    $f$ Riemann-integrierbar $\Leftrightarrow \lambda^1(\{x \in I \ | \ f \text{ ist nicht stetig in } x\}) = 0$\\
    In diesem Fall ist $f$ auch Lebesgue-integrierbar und die Integrale stimmen überein.
  \end{theorem}

  \begin{proof}
    siehe Aufschrieb
  \end{proof}

  \begin{theorem}
    $X$ metrischer Raum, $\mu$ Maß auf $Y$ und $f:X \times Y \to \mathbb{R}$ mit $f(x, \cdot)$ integrierbar bzgl. $\mu \ \forall x \in X$.\\
    Betrachte $F: X \to \mathbb{R}, F(x) = \int f(x,y) d\mu(y)$\\
    Sei $f(\cdot, y)$ stetig in $x_0 \in X$ für $\mu$-fast alle $y \in Y$. Weiter gebe es eine $\mu$-integrierbare Funktion $g: Y \to [0, \infty]$, so dass für alle $x \in X$ gilt: $|f(x,y)| \leq g(y) \ \forall y \in Y \setminus N_X$ mit einer $\mu$-Nullmenge $N_x$.\\
    Dann ist $F$ stetig in $x_0$.
  \end{theorem}

  \begin{proof}
    siehe Aufschrieb
  \end{proof}

   \sidenote{Vorlesung 14}{18.12.20}

  \begin{theorem}
    Sei $I \subseteq \mathbb{R}$ offenes Intervall, $\mu$ Maß auf $Y$ und $f: I \times Y \to \mathbb{R}$ mit $f(x, \cdot)$ integrierbar bzgl. $\mu$ für alle $x \in I$.\\
    Setze $F: U \to \mathbb{R}, F(x) = \int f(x,y) d\mu(y)$\\
    Es sei $f(\cdot, y)$ in $x_0$ differenzierbar für $\mu$-fast alle $y \in Y$ und es existiere $g: Y \to [0, \infty]$ $\mu$-integrierbar mit
    \begin{align*}
      \dfrac{|f(x,y) - f(x_0, y)|}{|x-x_0|} \leq g(y) \ \forall x\in I \ \forall y \in Y \setminus N_x
    \end{align*} 
    mit einer $\mu$-Nullmenge $N_x$. Dann folgt:
    \begin{align*}
      F'(x_0) = \int \dfrac{\partial f}{\partial x} (x_0, y) d\mu(y)
    \end{align*}
  \end{theorem}
  \begin{proof}
    siehe Aufschrieb
  \end{proof}

  \newpage

  \begin{lemma}
    $\script{U} \subseteq \mathbb{R}^n$ offen, $\mu$ Maß auf $Y$ und $f: \script{U} \times Y \to \mathbb{R}$ mit $f$ integrierbar bzgl. $\mu \ \forall x \in \script{U}$. Betrachte $F: \script{U} \to \mathbb{R}, F(x) = \int f(x,y) d\mu(y)$\\
    Es gebe eine $\mu$-Nullmenge $N \subseteq Y$, so dass $\forall y \in Y \setminus N$ gilt:
    \begin{align*}
      f(\cdot, y) \in C^1(\script{U}) \text{ und } |D_x f(x,y)| \leq g(y) \text{ mit } g: Y \to [0, \infty] \text{ integrierbar}
    \end{align*}
    $\implies F \in C^1(\script{U})$ und $\forall x \in \script{U}$ gilt:
    \begin{align*}
      \dfrac{\partial F}{\partial x_i}(x) = \int \dfrac{\partial f}{\partial x_i}(x,y) d\mu(y)
    \end{align*}
  \end{lemma}

  \begin{proof}
    siehe Aufschrieb
  \end{proof}

  \begin{example}
    \begin{align*}
      \int\limits_0^{\infty} \dfrac{\sin(x)}{x} dx = ? \ \ \ \ \text{Betrachte $F: [0, \infty] \to \mathbb{R}, F(t) = \int\limits_0^{\infty} e^{-tx} \dfrac{\sin{x}}{x} dx$}
    \end{align*}
    $f(t,x) := e^{-tx} \dfrac{\sin(x)}{x}$ hat für $t \geq \delta$ die Abschätzungen\\
    $|f(t,x)|, |\partial_t f(t,x)| \leq e^{-\delta x} =: g(x) \in L^1([0, \infty))$\\
    Lemma V.6 $\implies \forall t > 0$ gilt: 
    \begin{align*}
      F'(t) &= \int\limits_0^{\infty} e^{-tx} (-\sin{x}) dx\\
            &= [e^{-tx} \cos{x}]_{x=0}^{x=\infty} + t \int\limits_0^{\infty} e^{-tx} \cos{x} dx\\
            &= -1 + t^2 \int\limits_0^{\infty} e^{-tx} \sin{x} dx\\
            &= -1 - t^2 F'(t)
    \end{align*}
    $\implies F'(t) = \dfrac{-1}{1+t^2}$\\
    \newline
    ... (siehe Aufschrieb)\\
    \newline
    $\int\limits_0^{\infty} \dfrac{\sin(x)}{x} dx = \dfrac{\pi}{2}$
  \end{example}

  \newpage
  \begin{definition}[$L^p$-Norm]
    Für $\mu$-messbares $f: X \to \bar{\mathbb{R}}$ und $1 \leq p \leq \infty$ setzen wir
    \begin{align*}
      ||f||_{L^p(\mu)} := \begin{cases}
        (\int |f|^p d\mu)^{1/p} & \text{, für } 1\leq p < \infty\\
        \inf\{s>0 \ | \ \mu(\{|f| > s\})=0\} & \text{, für } p = \infty
      \end{cases}
    \end{align*}
    auf $\script{L}^p(\mu) = \{f:X \to \bar{\mathbb{R}} \ | \ f \mu-\text{messbar }, ||f||_{L^p(\mu)} < \infty\}$\\
    Betrachte Äquivalenzrelation $f\sim g \Leftrightarrow f(x) = g(x)$ für $\mu$-fast alle $x \in X$, und definiere den $\bm{L^p}$\textbf{-Raum} durch $\script{L}^p(\mu)/_{\sim}$.
  \end{definition}

  \begin{definition}
    Für $E \subseteq X$ messbar und $f: E \to \bar{\mathbb{R}}$ sei $f_0: X\to \bar{\mathbb{R}}$ die \textbf{Fortsetzung} mit $f_0(x)=0 \ \forall x \in X \setminus E$. Wir setzen dann
    \begin{align*}
      \script{L}^p(E) := \{f:E \to \bar{\mathbb{R}} \ | \ f_0 \in \script{L}^p(\mu)\}
    \end{align*}
    und $L^p(E,\mu) := \script{L}^p(E)/_{\sim}$.
  \end{definition}

  \begin{proposition}
    Für $1 \leq p \leq \infty$ ist $(L^p(\mu), ||\cdot||_{L^p(\mu)})$ ein normierter Vektorraum. Insbesondere gelten für $\lambda \in \mathbb{R}$ und $f,g \in L^p(\mu)$:
    \begin{enumerate}
      \item $||f||_{L^p} = 0 \implies f = 0$ $\mu$-fast überall
      \item $f \in L^p(\mu), \lambda \in \mathbb{R} \implies \lambda f \in L^p(\mu), \ ||\lambda f||_{L^p} = |\lambda| \ ||f||_{L^p}$
      \item $f,g \in L^p(\mu) \implies f+g \in L^p(\mu)$ und $||f+g||_{L^p} \leq ||f||_{L^p} + ||g||_{L^p}$
    \end{enumerate}
  \end{proposition}

  \begin{proof}
    siehe Aufschrieb
  \end{proof}

  \begin{lemma}[Youngsche Ungleichung]
    Für $1 < p,q < \infty$ mit $\dfrac{1}{p} + \dfrac{1}{q} = 1$ und $x,y \geq 0$ gilt: \ \ $xy \leq \dfrac{x^p}{p} + \dfrac{y^q}{q}$
  \end{lemma}

  \begin{proof}
    siehe Aufschrieb
  \end{proof}

  \newpage
  \begin{theorem}[Höldersche Ungleichung]
    Für $\mu$-messbare $f,g: X \to \bar{\mathbb{R}}$ gilt: \ \ $|\int fg d\mu| \leq ||f||_{L^p} ||g||_{L^p}$,\\
    falls $1 \leq p,q \leq \infty$ mit $\dfrac{1}{p} + \dfrac{1}{q} = 1$
  \end{theorem}

  \begin{proof}
    siehe Aufschrieb
  \end{proof}

  \begin{theorem}[Minkowski-Ungleichung]
    Für $f,g \in L^p(\mu)$ mit $1 \leq p \leq \infty$ gilt: \ \ $|| f+g ||_{L^p} \leq ||f||_{L^p} + ||g||_{L^p}$
  \end{theorem}

  \begin{proof}
    siehe Aufschrieb
  \end{proof}

  \begin{lemma}
    Sei $1 \leq p < \infty$ und $f_k = \sum\limits_{j=1}^k u_j$ mit $u_j \in L^p(\mu)$. Falls $\sum\limits_{j=1}^k ||u_j||_{L^p} < \infty$, so gelten:
    \begin{enumerate}[label=\roman*)]
      \item $\exists \ \mu$-Nullmenge $N$: $f(x) = \lim\limits_{k \to \infty} f_k(x) \ \forall x \in X \setminus N$ ex.
      \item mit $f := 0$ auf gilt $f \in L^p(\mu)$
      \item $||f - f_k||_{L^p} \to 0$ mit $k \to \infty$
    \end{enumerate}
  \end{lemma}

  \begin{proof}
    siehe Aufschrieb
  \end{proof}