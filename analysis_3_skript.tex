\documentclass[11pt,a4paper,fleqn,openany]{report}

\usepackage{analysis_3_sty}
%% Headers and Footers
\fancyhf[LH]{inoffizielles Skript\\von Lars Gavris}
\fancyhf[CH]{Analysis 3}
\fancyhf[RH]{SS20/21\\Prof. Lamm}

\begin{document}
  \sidenote{Vorlesung 1}{02.11.2020}
  % 
  % Vorwort
  % 
  \begin{goals}
    \begin{enumerate}
      \item[]
      \item Maßtheorie $\to$ Lebesgue-Maß\\(Volumen von Teilmengen des $\mathbb{R}^n$ bestimmen)
      \item Integralrechnung für Funktionen $f:\Omega \subseteq \mathbb{R}^n \to \mathbb{R}$\\$\to$ Lebesgue-Integrale (Satz von Fubini, ...)
      \item Version des Hauptsatzes $\to$ Satz von Gauß 
    \end{enumerate}
  \end{goals}

  \chapter{Maße und messbare Funktionen}

    \begin{notation}
      Menge $X$, Potenzmenge $\script{P}(X)$, eine Teilmenge von $\script{P}(X)$ heißt Mengensystem
    \end{notation}

    \begin{definition}
      Ein Mengensystem $\script{A} \subseteq \script{P}(X)$ heißt \textbf{$\bm{\sigma}$-Algebra}, falls:
      \begin{enumerate}[label=(\roman*)]
        \item $X \in \script{A}$
        \item $A \in \script{A} \implies X \setminus A \in \script{A}$
        \item $A_i \in \script{A}, \forall i \in \mathbb{N} \implies \bigcup\limits_{i \in \mathbb{N}} A_i \in \script{A}$
      \end{enumerate}
      Das Paar $(X, \script{A})$ heißt dann \textbf{messbarer Raum}.
    \end{definition}

    \begin{remark}
      \begin{enumerate}
        \item[]
        \item $A_i \in \script{A}, \forall i \in \mathbb{N} \implies \bigcap\limits_{i \in \mathbb{N}} A_i \in \script{A}$\\
              Denn: $\bigcap\limits_{i \in \mathbb{N}} A_i = X \setminus (\bigcup\limits_{i \in \mathbb{N}} X \setminus A_i)$
        \item $\emptyset = X \setminus X \in \script{A}$
        \item $A,B \in \script{A} \implies A \setminus B \in \script{A}$\\
              Denn: $A \setminus B = A \cap (X \setminus B)$
      \end{enumerate}
    \end{remark}

    \begin{example}
      \begin{enumerate}
        \item[]
        \item $\script{P}(X)$ ist $\sigma$-Algebra, $\{\emptyset, X\}$ ist $\sigma$-Algebra
        \item später: Menge aller messbaren Mengen eines äußeren Maßes bildet eine $\sigma$-Algebra.
      \end{enumerate}
    \end{example}
    
    \begin{theorem}
      Jeder Durchschnitt von (endlich oder unendlich vielen) $\sigma$-Algebren auf der selben Menge $X$ ist wieder eine $\sigma$-Algebra.
    \end{theorem}

    \begin{proof}
      $(A_i)_{i \in I}$ sei eine Familie von $\sigma$-Algebren bezüglich $X$.\\
      Offensichtlich gilt: $X \in \bigcap\limits_{i \in I} \script{A}_i$\\
      Sei $A \in \bigcap\limits_{i \in I} \script{A}_i \implies A \in \script{A}_i \ \forall i \in I \implies X \setminus A \in \script{A}_i \ \forall i \in I \implies X \setminus A \in \bigcap\limits_{i \in I} A_i$\\
      Analog für die abzählbare Vereinigung.
    \end{proof}

    \begin{definition}
      Für ein Mengensystem $\script{E} \subseteq \script{P}(X)$ heißt $\sigma(\script{E}) := \bigcap\{\script{A}|\script{A}$ ist $\sigma$-Algebra in $X$ mit $\script{E} \subseteq \script{A}\}$ die von $\script{E}$ \textbf{erzeugte $\bm{\sigma}$-Algebra}. Man nennt $\script{E}$ das \textbf{erzeugende System} von $\sigma(\script{E})$.
    \end{definition}

    \begin{remark}
      Dieser Durchschnitt ist nicht-trivial, denn $\script{P}(X)$ ist $\sigma$-Algebra mit $\script{E} \subseteq \script{P}(X)$.
    \end{remark}

    \begin{example}
      \begin{enumerate}
        \item[]
        \item Ist $E \subseteq X$ und $\script{E} = \{E\} \implies \sigma(\script{E}) = \{\emptyset, E, X \setminus E, X\}$
        \item Sei $(X,d)$ ein metrischer Raum. $\script{O} \subseteq \script{P}(X)$ sei das System der offenen Mengen. Die von $\script{O}$ erzeugte $\sigma$-Algebra heißt \textbf{Borel-$\bm{\sigma}$-Algebra} $\mathbb{B} (\script{O})=\mathbb{B}$. Ihre Elemente heißen \textbf{Borelmengen}.
        \item Seien $X \neq \emptyset$, $(Y, \script{C})$ messbarer Raum, $f:X \to Y$ eine Abbildung und das Urbild von $C \subseteq Y$: $f^{-1}(C) := \{x \in X|f(x) \in C\}$. Dann ist $f^{-1}(\script{C}):=\{f^{-1}(C)|C \in \script{C}\}$ eine $\sigma$-Algebra bzgl. $X$.\\
              Begründung: 
              \begin{itemize}
                \item $X \in f^{-1}(\script{C})$, denn $f^{-1}(Y) = X$ und $Y \in \script{C}$
                \item $f^{-1}(C) \in f^{-1}(\script{C}) \iff C \in \script{C}$,\\$f^{-1}(Y \setminus C) = X \setminus f^{-1}(C)$
                \item Erinnerung: $f^{-1}(A \cup B) = f^{-1}(A) \cup f^{-1}(B)$
              \end{itemize}
        \item Sei $X$ eine beliebige Menge und $\script(E)_i \subseteq \script{P}(X)$, $i \in I$, Mengensysteme, dann gilt: $\sigma(\bigcup\limits_{i \in I} \script{E}_i) = \sigma(\bigcup\limits_{i \in I} \sigma(\script{E}_i))$\\
              Begründung:
              \begin{itemize}
                \item Klar: $\subseteq$
                \item Andererseits enthält $\sigma(\bigcup\limits_{i \in I} \script{E}_i)$ das System $\bigcup\limits_{i \in I} \sigma (\script{E}_i)$ und ist eine $\sigma$-Algebra $\implies \sigma(\bigcup\limits_{i \in I} \sigma(\script{E}_i)) \subseteq \sigma(\bigcup\limits_{i \in I} \script{E}_i)$
              \end{itemize}
      \end{enumerate}
    \end{example}

    \begin{notation}
      $\bar{\mathbb{R}} := \mathbb{R} \cup \{+\infty, -\infty\}$ mit $-\infty < a < +\infty,\ \forall a \in \mathbb{R}$
    \end{notation}

    \newpage

    \begin{definition}
      Eine Folge $(s_k) \subseteq \bar{\mathbb{R}}\ (k \in \mathbb{N})$ konvergiert gegen $s \in \bar{\mathbb{R}}$, falls eine der folgenden Alternativen gilt:
      \begin{enumerate}[label=(\roman*)]
        \item $s \in \mathbb{R}$ und $\forall \epsilon>0$ gilt: $s_k \in (s-\epsilon, s+\epsilon) \subseteq \mathbb{R}$ für k hinreichend groß
        \item $s=\infty$ und $\forall r \in \mathbb{R}: s_k \in (r, \infty]$ für k hinreichend groß
        \item $s=-\infty$ und $\forall r \in \mathbb{R}: s_k \in [-\infty, r)$ für k hinreichend groß
      \end{enumerate}
      $(s_k) \subseteq \mathbb{R}$ ist genau dann in $\bar{\mathbb{R}}$ konvergent, wenn sie entweder in $\mathbb{R}$ konvergiert, oder bestimmt gegen $\pm\infty$ divergiert.
    \end{definition}

    \begin{example}
      \begin{itemize}
        \item[]
        \item $s_k$ monoton $\implies\ s_k$ konvergiert in $\bar{\mathbb{R}}$
        \item $a_k \geq 0 \implies \sum\limits_{k \in \mathbb{N}}a_k \in \bar{\mathbb{R}}$
        \item Eine Menge $U \subseteq \bar{\mathbb{R}}$ ist genau dann offen, wenn $U \cap \mathbb{R}$ offen ist und im Fall $+\infty \in U$ (bzw. $-\infty \in U$) ein $a \in \mathbb{R}$ existiert, sodass $(a,\infty] \subseteq U$ (bzw. $[-\infty,a) \subset U$) ist.
        \item Die Borel-$\sigma$-Algebra $\bar{\mathbb{B}}$ auf $\bar{\mathbb{R}}$ wird durch die offenen Mengen in $\bar{\mathbb{R}}$ erzeugt. Es gilt: $\bar{\mathbb{B}} = \{B \cup E|B \in \mathbb{B}, E \subseteq \{-\infty,+\infty\}\}$ 
      \end{itemize}
    \end{example}

    \begin{notation}
      \underline{Addition:}
      \begin{tabular}[t]{c | c c c}
        + & $-\infty$ & $\mathbb{R}$ & $+\infty$\\
        \hline
        $-\infty$ & $-\infty$ & $-\infty$ & /\\
        $\mathbb{R}$ & $-\infty$ & $\mathbb{R}$ & $+\infty$\\
        $+\infty$ & / & $+\infty$ & $+\infty$
      \end{tabular}\\
      $\sup\emptyset := -\infty,\ \inf\emptyset := +\infty$ konsistent mit $A,B \subseteq \mathbb{R}$ gilt $A \subseteq B \implies \sup A < \sup B$ und $\inf A \geq \inf B$ 
    \end{notation}

    \begin{definition}
      Sei $\script{A} \subseteq \script{P}(X)$ eine $\sigma$-Algebra, eine nicht-negative Mengenfunktion $\mu: \script{A} \rightarrow [0, \infty]$ heißt \textbf{Maß} auf $\script{A}$, falls:
      \begin{enumerate}[label=(\roman*)]
        \item $\mu(\emptyset) = 0$
        \item für beliebige paarweiße disjunkte $A_i \in \script{A}$, $i \in \mathbb{N}$, gilt:\\
              $\mu(\bigcup\limits_{i \in \mathbb{N}} A_i) = \sum\limits_{i \in \mathbb{N}} \mu (A_i)$ \hfill ($\sigma$-Additivität)
      \end{enumerate}
      Das Tripel $(X, \script{A}, \mu)$ heißt \textbf{Maßraum}.
    \end{definition}

    \begin{remark}
      \begin{enumerate}
        \item[]
        \item Für endlich viele paarweiße disjunkte $A_i \in \script{A}$, $i=1,...,n$, folgt aus (ii) indem man $A_i=\emptyset$ für $i=n+1, ...$ setzt: $\mu(\bigcup\limits_{i=1}^n A_i) = \sum\limits_{i=1}^n \mu(A_i)$
        \item Monotonie des Maßes: $A,B \in \script{A}$ mit $A \subseteq B \implies \mu(A) \leq \mu(B) = \mu(A \cup (B \setminus A)) = \mu(A) + \mu(B \setminus A)$
      \end{enumerate}
    \end{remark}

    \begin{definition}
      Sei $(X, \script{A}, \mu)$ ein Maßraum. Das Maß $\mu$ heißt \textbf{endlich}, wenn $\mu(A) < \infty\ \forall A \in \script{A}$ und $\bm{\sigma}$\textbf{-endlich}, wenn es eine Folge $(X_i) \in \script{A}$ mit $\mu(X_i) < \infty$ gibt, sodass $X=\bigcup\limits_{i \in \mathbb{N}} X_i$. Falls $\mu(X) = 1$, so wird $\mu$ \textbf{Wahrscheinlichkeits-Maß} genannt.
    \end{definition}

    \begin{example}
      \begin{enumerate}
        \item[]
        \item Sei $X$ eine beliebige Menge, $\script{A} = \script{P}(X)$, für $x \in X$ sei $\delta_x(A) := \begin{cases}1, & x \in A\\0, & x \notin A\end{cases}$ (\textbf{Dirac-Maß})
              \begin{itemize}
                \item Es gilt $\delta_x(A) \in \{0,1\}$, $\delta_x(\emptyset) = 0$, $\delta_x(X) = 1$.
                \item Sei $A = \bigcup\limits_{k \in \mathbb{N}} A_k$ gegeben mit $A_k$ paarweiße disjunkt und $x \in A \implies x \in A_k$ für genau ein $k \in \mathbb{N} \implies\ \sigma$-Additivität.
                \item Für $x \notin A$ gilt sowieso $\delta_x{A} = 0$
              \end{itemize}
              $\implies$ Das Dirac-Maß ist ein Wahrscheinlichkeits-Maß
        \item \textbf{Zählmaß:} $X$ beliebige Menge \sidenote{Vorlesung 2}{06.11.2020}\\
              $card: \script{P}(X) \to [0, \infty]$\\
              $card(A) := \begin{cases}\text{Anzahl der Elemente von A}, & \text{falls A endlich}\\\infty, sonst\end{cases}$\\
              Für $A = \bigcup\limits_{k\in \mathbb{N}}A_k$ endlich und paarweiße disjunkt ist die $\sigma$-Additivität klar.\\
              \\
              Sei $A$ unendlich und $A = \bigcup\limits_{k\in \mathbb{N}}A_k$.
              \begin{enumerate}
                \item nur endlich viele $A_k$ nicht-trivial\\
                      $\implies \exists k_0: A_{k_0}$ ist unendlich
                \item abzählbar viele $A_k$ sind nicht-trivial $\implies$ Behauptung
              \end{enumerate}
              $\implies$ Behauptung\\
              \\
              Zählmaß ist $\sigma$-endlich $\Leftrightarrow X$ ist abzählbar\\
              Zählmaß ist endlich $\Leftrightarrow X$ ist endlich
      \end{enumerate}
    \end{example}

    \begin{example}
      $X$ beliebige Menge, $\script{A} \subseteq \script{P}(X)$ $\sigma$-Algebra, $\mu(A)=0 \ \forall A \in \script{A}$
    \end{example}

    \begin{theorem}[Stetigkeitseigenschaften von Maßen]
      Sei $(X,\script{A},\mu)$ Maßraum. Dann gelten für Mengen $A_i \in \script{A}, i \in \mathbb{N}$ folgende Aussagen:

      \begin{enumerate}[label=(\roman*)]
        \item Aus $A_1 \subseteq A_2 \subseteq A_3 \subseteq ...$ folgt: $\mu (\bigcup\limits_{i \in \mathbb{N}} A_i) = \lim\limits_{i \to \infty} \mu (A_i)$
        \item Aus $A_1 \supseteq A_2 \supseteq A_3 \supseteq ...$ mit $\mu(A_1)<\infty$, folgt: $\mu (\bigcap\limits_{i \in \mathbb{N}} A_i) = \lim\limits_{i\to \infty} \mu (A_i)$
        \item $\mu(\bigcup\limits_{i\in \mathbb{N}} A_i) \leq \sum\limits_{i\in \mathbb{N}} \mu(A_i)$
      \end{enumerate}
    \end{theorem}

    \begin{remark}
      \begin{enumerate}
        \item[]
        \item \begin{enumerate}[label=(\roman*)]
          \item Stetigkeit von unten
          \item Stetigkeit von oben
          \item $\sigma$-Subadditivität von $\mu$
        \end{enumerate}
        \item Bedingung $\mu(A_i) \leq \infty$ in (ii) kann durch $\mu(A_k) \leq \infty$ für ein $k \in \mathbb{N}$ ersetzt werden, kann aber nicht weggelassen werden.\\
              Begründung:\\
              $A_k = {k, k+1,...} \subseteq \mathbb{N}$\\
              $card(A_k) = \infty \ \forall k \in \mathbb{N}$\\
              Aber: $card(\bigcap\limits_{i \in \mathbb{N}} A_i) = card(\emptyset) = 0$ 
      \end{enumerate}
    \end{remark}

    \begin{proof}
      \begin{enumerate}[label=(\roman*)]
        \item[]
        \item $\tilde{A_1} := A_1, \ \tilde{A_k} := A_k \setminus A_{k-1}, \ k \geq 2$\\
              $\tilde{A_i}$ sind paarweiße disjunkt.\\
              $\bigcup\limits_{i \in \mathbb{N}} \tilde{A_i} = \bigcup\limits_{i \in \mathbb{N}} A_i$\\
              $\mu(\bigcup\limits_{i \in \mathbb{N}} A_i) = \mu(\bigcup\limits_{i \in \mathbb{N}} \tilde{A_i}) = \sum\limits_{i \in \mathbb{N}} \mu(\tilde{A_i}) = \lim\limits_{k \to \infty}(\sum\limits_{i=1}^k \mu(\tilde{A_i})) = \lim\limits_{k \to \infty} \mu(\bigcup\limits_{i=1}^k A_k) = \lim\limits_{k \to \infty} \mu(A_k)$
        \item $A'_k := A_1 \setminus A_k \ \implies \ A'_1 \subseteq A'_2 \subseteq ...$\\
              Es gilt: $\mu(A_1) = \mu(A_1 \cap A_k) + \mu(A_1 \setminus A_k) = \mu(A_k) + \mu(A'_k)$\\
              $\implies \mu(A_1) - \lim\limits_{k \to \infty} \mu(A_k) = \lim\limits_{k \to \infty} \mu(A'_k) \stackrel{(i)}{=} \mu(\bigcup\limits_{k \in \mathbb{N}} A'_i) = \mu(A_1 \setminus \bigcap\limits_{i \in \mathbb{N}})\\ = \mu(A_1) - \mu(\bigcap\limits_{i \in \mathbb{N}} A_i)$
        \item Es genügt, die Folge $B_1 = A_1,\ B_i \stackrel{i \geq 2}{=} A_i \setminus \bigcup\limits_{j=1}^{i-1}A_j$ zu betrachten.\\
              $\bigcup\limits_{i \in \mathbb{N}} A_i = \bigcup\limits_{i \in \mathbb{N}} B_i$ und $(B_i)$ ist paarweiße disjunkt.\\
              $\implies \mu(\bigcup\limits_{i \in \mathbb{N}} A_i) = \mu(\bigcup\limits_{i \in \mathbb{N}} B_i) = \sum\limits_{i \in \mathbb{N}} \mu(B_i) \leq \sum\limits_{i \in \mathbb{N}} \mu(A_i)$
      \end{enumerate}
    \end{proof}

    \begin{definition}
      $(X, \script{A}, \mu)$ Maßraum.\\
      Jede Menge $A \in \script{A}$ mit $\mu(A) = 0$ heißt $\bm{\mu}$\textbf{-Nullmenge}. Das System aller $\mu$-Nullmengen bezeichnen wir mit $\bm{\script{N}(\mu)}$. Das Maß $\mu$ heißt \textbf{vollständig}, wenn gilt:\\
      \[
        N \subseteq A \text{ für ein } H \in \script{A} \text{ mit } \mu(A)=0\\
        \implies N \in \script{A} \text{ und } \mu(N)=0
      \]
    \end{definition}

    \begin{remark}
      Nicht jedes Maß ist vollständig:
      \[
        \script{A} \neq \script{P}(X) \ \mu(A) = 0 \ \forall A \in \script{A}
      \]
      Allerdings lässt sich jedes Maß vervollständigen:\\
      Sei $(X, \script{A}, \mu)$ Maßraum und $\script{T}_{\mu}$ sei das System aller Mengen $N \subseteq X$ für die keine $\mu$-Nullmenge $B \in \script{N}(\mu)$ existiert mit $N \subseteq B$. Es gilt:\\
      \[
        \mu \text{ vollständig } \Leftrightarrow \script{T}_{\mu} \subseteq \script{A}
      \]
      Definiere auf $\bar{A_{\mu}} := \{A \cup N | A \in \script{A}, N \in \script{T}_{\mu}\}$ die Mengenfunktion $\bar{\mu}$ durch $\bar{\mu}(A\cup N) := \mu(A) \ \forall A \in \script{A}, N\in \script{T}_{\mu}$
    \end{remark}

    \begin{remark}
      $\bar{\mu}$ ist wohldefiniert: $A \cup N = B \cup P$ mit $A,B\in \script{A}, \ P,N \in \script{T}_{\mu} \implies \exists C \in \script{A}, \mu(C) = 0: P \subseteq C \implies A \subseteq B \cup C \implies \mu(A) \leq \mu(B) + \mu(C) = \mu(B)$\\
      Symm $\implies \mu(A) = \mu(B)$\\
      $\bar{\mu}$ heißt \textbf{Vervollständigung} von $\mu$
    \end{remark}

    \begin{theorem}
      $(X,\script{A}, \mu)$ Maßraum. Dann ist $\bar{\script{A}}_{\mu}$ eine $\sigma$-Algebra und $\bar{\mu}$ ein vollständiges Maß auf $\bar{\script{A}}_{\mu}$, welches mit $\mu$ auf $\script{A}$ übereinstimmt.
    \end{theorem}

    \begin{proof}
      Offensichtlich:
      \begin{enumerate}
        \item $\script{A} \subseteq \bar{\script{A}}_{\mu}$
        \item $\script{T}_{\mu}$ ist abgeschlossen unter Abz. $\bigcup$
      \end{enumerate}
      $\script{A}$ ist auch abgeschlossen unter abzählbarer Vereinigung\\
      $\implies \ \bar{\script{A}}_{\mu}$ abgeschlossen unter abzählbarer Vereinigung\\
      Sei $x \in \bar{\script{A}}_{\mu}$. Für $E \in \bar{\script{A}}_{\mu}$ ex. ein $A \in \script{A}, \ N \in \script{T}_{\mu}$ und $B \in \script{A}$ und $N \subseteq B$ mit $\mu(B) = 0$, sodass $E = A \cup N \\ \implies B \setminus N \in \script{T}_{\mu} \\ \implies X \setminus E = (X \setminus (A\cup B)) \cup (B \setminus N) \in \bar{\script{A}}_{\mu} \\ \implies \bar{\script{A}}_{\mu}$ ist $\sigma$-Algebra \\
      $\bar{\mu}$ ist Maß (ist klar)\\
      Sei $M \subseteq B = A \cup N$ mit $A \in \script{A}, N \in \script{T}_{\mu}$ und $\bar{\mu}(B) = \mu(A) = 0$ \\
      Aus $M = (M \cap A) \cup (M \cap N) \in \script{T}_{mu} \cup \script{T}_{\mu} = \script{T}_{\mu} \in \bar{\script{A}}_{\mu} \\ \implies \ \bar{\mu}$ ist vollständig.  
    \end{proof}

    \begin{theorem}
      $(X, \script{A}, \mu)$ Maßraum und $(X, \bar{\script{A}}_{\mu}, \bar{\mu})$ sei Vervollständigung. Ferner sei $(X, \script{B}, \nu)$ ein vollständiger Maßraum mit $\script{A} \subseteq \script{B}$ und $\mu = \nu$ auf $\script{A}$. Dann ist $\bar{\script{A}}_{\mu} \subseteq \script{B}$ und $\bar{\mu} = \nu$ auf $\bar{\script{A}}_{\mu}$. 
    \end{theorem}

    \begin{proof}
      Aus $\script{A} \subseteq \script{B}$ und $\mu = \nu$ auf $\script{A}$ folgt: $\script{N}(\mu) \subseteq \script{N}(\nu) \implies \script{T}_{\mu} \subseteq \script{T}_{\mu}$\\
      $\nu$ vollständig $\implies \ \script{T}_{\nu} \subseteq \script{B} \implies \script{T}_{\mu} \subseteq \script{B} \implies \bar{\script{A}}_{\mu} \subseteq \script{B}$\\
      Da $\bar{\mu}$ auf $\bar{\script{A}}_{\mu}$ vollständig durch $\mu$ auf $\script{A}$ bestimmt ist, folgt sofort $\bar{\mu} = \nu$ auf $\bar{\script{A}}_{\mu}$, da $\mu = \nu$ auf $\script{A}$.
    \end{proof}

    \begin{definition}
      $(X, \script{A}), (Y, \script{C})$ messbare Räume. Eine Abbildung $f: X \to Y$ heißt $\bm{\script{A}-\script{C}-}$\textbf{messbar}, falls $f^{-1}(\script{C}) \subseteq{\script{A}}$
    \end{definition}

    \begin{notation}
      Falls $\script{A}, \script{C}$ klar sind, bezeichnen wir $f$ einfach als messbar. 
    \end{notation}

    \begin{example}
      \begin{enumerate}
        \item[]
        \item $(X, \script{A}), (Y, \script{C})$ beliebige messbare Räume.\\
              Sei $y_0 \in Y$ und $f: X \to Y, \ f(x) = y_0 \ \forall x \in X$\\
              $\implies f$ ist $\script{A}$-$\script{C}-$messbar
        \item $\chi_R: X \to \mathbb{R}, \chi_R(x) = \begin{cases} 1 \text{, falls } x \in E \subseteq X \\ 0 \text{, sonst}\end{cases}$\\
              $\mathbb{R}$ wird versehen mit Borel-$\sigma$-Algebra $\script{B}$. Für $(X, \script{A})$ messbarer Raum gilt:\\
              $\chi_R \ \script{A}$-$\script{B}$-messbar $\Leftrightarrow \ E \in \script{A}$
        \item Komposition messbarer Abbildungen ist messbar.\\
              $(X, \script{A}), (Y, \script{C}), (Z, \script{D})$ messbare Räume.\\
              $f:X  \to Y$ $\script{A}$-$\script{C}$-messbar\\
              $g:Y \to Z$ $\script{C}$-$\script{D}$-messbar\\
              $\implies g \circ f: X \to Z$ ist $\script{A}$-$\script{D}$-messbar, denn:\\
              $(g \circ f)^{-1}(\script{D}) = f^{-1}(g^{-1}(\script{D})) \subseteq f^{-1}(\script{C}) \subseteq \script{A}$
      \end{enumerate}
    \end{example}

    \begin{lemma}
      $(X, \script{A}), (Y, \script{C})$ messbare Räume und $\script{C} := \sigma(\script{E})$. Jede Abbildung $f: X \to Y$ mit $f^{-1}(\script{E}) \subseteq \script{A}$ ist $\script{A}$-$\script{C}$-messbar.
    \end{lemma}

    \begin{proof}
      Es gilt: $f^{-1}(\script{C}) = f^{-1}(\sigma(\script{E})) \stackrel{s. Blatt 1}{=} \sigma(f^{-1}(\script{E})) \subseteq \sigma(\script{A}) = \script{A}$
    \end{proof}

    \begin{example}
      \begin{enumerate}
        \item[]
        \item Jede stetige Abbildung $f: \mathbb{R}^n \to \mathbb{R}^n$ ist $\mathbb{B}^n$-$\mathbb{B}^n$-messbar\\
              (man sagt: $f$ ist \textbf{borel-messbar}).\\
              Denn $\mathbb{B}^n = \sigma(\{\text{offene Teilmengen des } \mathbb{R}^n\})$ und Urbilder offener Mengen sind offen für $f$ stetig (siehe. Ana 1)
        \item Sei $X \neq \emptyset$ Menge, $(Y, \script{C})$ messbarer Raum, $f: X \to Y$ Abbildung.\\
              Nach Bsp. aus 1. Vorlesung ist $f^{-1}(\script{C})$ $\sigma$-Algebra.\\
              Offensichtlich ist $f^{-1}(\script{C}) \subseteq \script{P}(X)$ die kleinste $\sigma$-Algebra und $f$ messbar.
      \end{enumerate}
    \end{example}

    \begin{notation}
      Multiplikation und Division in $\bar{\mathbb{R}} = \mathbb{R} \cup \{\pm \infty\}$\\
      $s * (\pm \infty) = (\pm \infty) * s = \begin{cases}\pm \infty & \text{, falls } s \in (0, \infty] \\ 0 & \text{, falls } s = 0 \\ \mp \infty & \text{, falls } s \in [-\infty, 0)\end{cases}$\\
      $\dfrac{1}{t} = 0$ für $t = \pm \infty$
    \end{notation}

    \begin{definition}
      $(X, \script{A})$ messbarer Raum und $D \in \script{A}$.\\
      Eine Funktion $f: D \to \bar{\mathbb{R}}$ heißt \textbf{numerische Funktion}.
    \end{definition}

    \begin{lemma}
      $(X, \script{A})$ messbarer Raum, $D \in \script{A}$ und $f: D \to \bar{\mathbb{R}}$.\\
      Dann sind folgende Aussagen äquivalent:
      \begin{enumerate}[label=(\roman*)]
        \item $f$ ist $\script{A}$-$\bar{\mathbb{B}}^1$-messbar
        \item $\forall \ \script{U} \subseteq \mathbb{R}$ offen ist $f^{-1}(\script{U}) \in \script{A}$ und $f^{-1}(\{\infty\}), f^{-1}(\{-\infty\}) \in \script{A}$
        \item $\{f \leq s\} := \{x \in D \ |\ f(x) \in [-\infty, s]\} \in \script{A} \ \forall s \in \mathbb{R}$
        \item $\{f < s\} := \{x \in D \ |\ f(x) \in [-\infty, s)\} \in \script{A} \ \forall s \in \mathbb{R}$
        \item $\{f \geq s\} := \{x \in D \ |\ f(x) \in [s, \infty]\} \in \script{A} \ \forall s \in \mathbb{R}$
        \item $\{f > s\} := \{x \in D \ |\ f(x) \in (s, \infty]\} \in \script{A} \ \forall s \in \mathbb{R}$
      \end{enumerate}
    \end{lemma}

    \begin{proof}
      $\bar{\mathbb{B}}^1$ wird erzeugt durch die offenen Mengen und $\pm \infty \implies$ (i) $\Leftrightarrow$ (ii)\\
      \\
      (iii) $\Leftrightarrow$ (iv) $\Leftrightarrow$ (v) $\Leftrightarrow$ (vi) denn:\\
      (iv) $\implies$ (iii): ${f \leq s} = \bigcap\limits_{k \in \mathbb{N}} \{f < s + \dfrac{1}{k}\}$\\
      (iii) $\implies$ (vi): $\{f > s\} = D \setminus \{f \leq s\}$\\
      (vi) $\implies$ (v): $\{f \geq \bigcap\limits_{k \in \mathbb{N}} \{f > s - \dfrac{1}{k}\}\}$\\
      (v) $\implies$ (iv): $\{f < s\} = D \setminus \{f \geq s\}$\\
      \\
      (ii) $\implies$ (vi), denn: $\{f > s\} = f^{-1}(s,\infty) \cup f^{-1}(\{\infty\}) \in \script{A}$\\
      \\
      Für ein offenes Intervall $(a,b)$ gilt: $f^{-1}((a,b)) = \{f > a\} \cap \{f < b\} \in \script{A}$\\
      Eine der Aussagen (und damit alle) (iii) - (vi) gelte.\\
      Mann kann zeigen: Jede offene Menge $U \subseteq \mathbb{R}$ lässt sich als abzählbare Vereinigung $\script{U} = \bigcup\limits_{k \in \mathbb{N}} I_k$ von offenen Intervallen $I_k = (a_k, b_k)$ schreiben (siehe Blatt 2).\\
      $\implies f^{-1}(\script{U}) = \bigcup\limits_{k \in \mathbb{N}} f^{-1}(I_k) \in \script{A}$\\
      $f^{-1}(\{\infty\}) = \bigcap\limits_{k \in \mathbb{N}} \{f > k\} \in \script{A}, \ \ \ f^{-1}(\{-\infty\}) = \bigcap\limits_{k \in \mathbb{N}} \{f < -k\} \in \script{A} \ \ \ \implies$ (ii)
    \end{proof}

    \begin{remark}
      In (iii) - (vi) reicht es aus, $s \in \mathbb{Q}$, statt $s \in \mathbb{R}$ zu haben, denn es gilt z.B.:\\
      $\{f \geq s\} = \bigcap\limits_{\stackrel{q \in \mathbb{Q}}{s > q}} \{f > q\}$
    \end{remark}

    \sidenote{Vorlesung 3}{09.11.20}

    \begin{lemma}
      Sei $(X, \script{A})$ ein messbarer Raum, $D \in \script{A}$ und $f,g: D \to \bar{\mathbb{R}}$ $\script{A}$-messbar. Dann sind die Mengen $\{f < g\} := \{x \in D: f(x) < g(x)\}$ und $\{f \leq g\} := \{x \in D: f(x) \leq g(x)\}$ Elemente aus $\script{A}$.
    \end{lemma}

    \begin{proof}
      Es gilt: $\{f < g\} = \bigcup\limits_{q \in \mathbb{Q}} (\{f < g\} \cap \{g > q\}) \in \script{A}$, denn:\\
      $\{f < g\}, \{g > q\} \in \script{A}$ (s. Lemma I.14)\\
      $\{f \leq g\} = D \setminus \{f > g\} \in \script{A}$
    \end{proof}

    \begin{remark}
      Im folgenden Satz sind die Grenzfunktionen paarweiße definiert, z.B.:\\
      $\liminf f_x : X \to \bar{\mathbb{R}}$ ist definiert durch: \ \ $(\liminf\limits_{k \to \infty} f_k)(x) := \liminf\limits_{k \to \infty} f_k (x)$
    \end{remark}

    \begin{theorem}
      $(X, \script{A})$ messbarer Raum, $D \in \script{A}$ und $f_k:D \to \bar{\mathbb{R}}$ Folge von $\script{A}$-messbaren Funktionen.\\
      Dann sind auch folgende Funktionen $\script{A}$-messbar:
      \begin{center}
        $\inf\limits_{k \in \mathbb{N}} f_k, \ \sup\limits_{k \in \mathbb{N}} f_k, \ \liminf\limits_{k \to \infty} f_k, \ \limsup\limits_{k \to \infty} f_k$
      \end{center}
    \end{theorem}

    \begin{proof}
      Für $s \in \mathbb{R}$ gilt:\\
      $\{\inf\limits_k f_k \geq s\} = \bigcap\limits_{k \in \mathbb{N}} \{f_k \geq s\} \in \script{A}$, denn nach Lemma I.14 ist $\{f_k \geq s\} \in \script{A}$\\
      $\{\sup\limits_k f_k \leq s\} = \bigcap\limits_{k \in \mathbb{N}} \{f_k \leq s\} \in \script{A}$\\
      $\stackrel{\text{Lemma I.14}}{\implies} \inf f_k, \sup f_k$ sind $\script{A}$-messbar\\
      $\liminf\limits_{k \to \infty} f_k = \sup\limits_{k \in \mathbb{N}} (\inf\limits_{l \geq k} f_l)$ ist $\script{A}$-messbar.\\
      $\limsup\limits_{k \to \infty} f_k = \inf\limits_{k \in \mathbb{N}} (\sup\limits_{l \geq k} f_l)$ ist $\script{A}$-messbar.
    \end{proof}

    \begin{notation}
      Seien $D \in \script{A}$ und $f: D \to \bar{\mathbb{R}}$, dann sind $f^{\pm}:D \to [0, \infty]$ definiert durch:\\
      $f^+ := max(f, 0) \geq 0$ und $f^- := max(-f, 0) = -min(f, 0) \geq 0$\\
      $\implies f = f^+ - f^-, \ |f| = f^+ + f^-$
    \end{notation}

    \begin{theorem}
      $(X, \script{A})$ messbarer Raum, $D \in \script{A}$, $f,g: D \to \bar{\mathbb{R}}$ $\script{A}$-messbar, $\alpha \in \mathbb{R}$.\\
      Dann sind die Funktionen
      \begin{center}
        $f+g, \ \alpha f, \ f^{\pm}, \ max(f,g), \ min(f,g), \ |f|, \ fg, \ \dfrac{f}{g}$
      \end{center}
      auf ihren Definitionsbereichen, die in $\script{A}$ liegen $\script{A}$-messbar.
    \end{theorem}

    \begin{proof}
      \begin{enumerate}
        \item[]
        \item \underline{$f,g:D \to \mathbb{R}$}
          \begin{itemize}
            \item $\{f+g < t\} = \bigcup\limits_{\stackrel{r,s \in \mathbb{Q}}{r+s<t}} \{f < r\} \cap \{g < s\} \in \script{A}$\\
              $\{-f < t\} = \{f > -t\} \in \script{A}$\\
              $\implies f+g, -f \script{A}$-messbar. Ebenso $\alpha f$ für $\alpha \in \mathbb{R}$
            \item Für $\script{C} \in C^{\infty} (\mathbb{R})$ ist $\script{C} \circ f$ messbar, denn für $\script{U} \subseteq \mathbb{R}$ offen ist $\script{C}^{-1}(\script{U})$ offen und damit $(\script{C} \circ f)^{-1}(\script{U}) = f^{-1}(\script{C}^{-1}(\script{U})) \in \script{A}$\\
              $\implies f^{\pm}$ sind $\script{A}$-messbar (wähle $\script{C}(s)) = max(\pm s,0))$\\
              $\implies |f| = f^+ + f^-$,\\
              $max(f,g) = \dfrac{1}{2} (f + g + |f-g|)$ und\\
              $min(f,g) = \dfrac{1}{2} (f + g - |f-g|)$ sind $\script{A}$-messbar
            \item $f^2 = \script{C} \circ f$ mit $\script{C}(s) = s^2$ und\\
              $fg = \dfrac{1}{4} ((f+g)^2 - (f-g)^2)$ $\script{A}$-messbar
            \item $\dfrac{1}{g}$ ist $\script{A}$-messbar, denn:\\
              $\{\dfrac{1}{g} < s\} = \begin{cases} 
                \{\dfrac{1}{s} < g < 0\} & , s < 0 \\ 
                \{g < 0\} & s = 0 \\
                \{g < 0\} \cup \{g > \dfrac{1}{2}\} & s > 0
              \end{cases}$
          \end{itemize}
        \item \underline{$f,g$ beliebig}\\
          Betrachte $f_k(x) = \begin{cases}
            k & , f(x) \geq k\\
            -k & , f(x) \leq -k\\
            f(x) & , \text{ sonst}
          \end{cases} \in \mathbb{R}$\\
          Analog $g_k(x)$. $f_k, g_k$ sind $\script{A}$-messbar $\forall k$\\
          Punktweise gilt: $f_k(x) \to f(x), g_k(x) \to g(x)$\\
          Ebenso: $f_k+g_k \to f+g, \alpha f_k \to \alpha f, ... , f_k g_k \to fg$ punktweise.\\
      \end{enumerate}
      Der Allgemeine Fall folgt aus 1. und Satz I.16.
    \end{proof}

    \newpage

    \begin{notation}
      Sei $(X, \script{A}, \mu)$ Maßraum. Man sagt, die Aussage $A[x]$ ist wahr \textbf{für $\bm{\mu}$-fast alle} $x \in M \in \script{A}$ oder \textbf{$\bm{\mu}$-fast überall} auf M, falls es eine $\mu$-Nullmenge $N$ gibt mit
      \begin{center}
        $\{x \in M: A[x] \text{ ist falsch}\} \subseteq N$
      \end{center}
      Dabei wird nicht verlangt, dass $\{x \in M: A[x] \text{ ist falsch}\}$ selbst zu $\script{A}$ gehört.\\
      Zum Beispiel bedeutet für Funktionen $f,g: X \to \bar{\mathbb{R}}$ die Aussage \glqq$f(x) \leq g(x)$ für $\mu$-fast alle $x \in X$\grqq, dass es eine Nullmenge $N$ gibt, so dass $\forall x \in X \setminus N$ gilt: $f(x) \leq g(x)$.\\
      Eine Funktion $h$ ist \glqq$\mu$-fast überall auf $X$ definiert\grqq, wenn $h$ auf $D \in \script{A}$ definiert ist und $\mu(X \setminus D) = 0$. 
    \end{notation}

    \begin{example}[\glqq Konvergenz $\mu$-fast überall \grqq]
      Eine Folge von Funktionen $f_k:D \to \bar{\mathbb{R}}$ konvergiert punktweise $\mu$-fast überall gegen $f: D \to \bar{\mathbb{R}}$, wenn es eine $\mu$-Nullmenge $N$ gibt, so dass $\forall x \in D \setminus N$ gilt:
      \begin{center}
        $\lim\limits_{k \to \infty} f_k(x) = f(x)$
      \end{center}
    \end{example}

    \begin{goal}
      Messbarkeit für Funktionen, die nur $\mu$-fast überall definiert sind.
    \end{goal}

    \begin{definition}
      $(X, \script{A}, \mu)$ Maßraum. Eine auf $D \in \script{A}$ definierte Funktion $f: D \to \bar{\mathbb{R}}$ heißt $\mu$-messbar (auf $X$), wenn $\mu(X \setminus D) = 0$ und $f$ $\script{A|_D}$-messbar ist.\\
      ($\script{A}|_D := \{A \cap D | A \in \script{A}\}$, siehe Blatt 1)
    \end{definition}

    \begin{remark}
      \begin{enumerate}
        \item[]
        \item Unterscheiden zwischen $\script{A}$-messbaren Funktionen (auf $X$), die \underline{überall} auf $X$ definiert sind, und $\mu$-messbaren Funktionen (auf $X$), die in der Regel nur \underline{$\mu$-fast überall} definiert sind.
        \item Analog zu $\script{A}$-Messbarkeit verwenden wir $\mu$-Messbarkeit auf für Funktionen, die nur auf Teilmengen definiert sind:\\
          Sei $(X, \script{A}, \mu)$ Maßraum, $D \in \script{A}$. $f: E \to \bar{\mathbb{R}}$ heißt \textbf{$\bm{\mu}$-messbar} (auf $D$), wenn $E \subseteq D$ in $\script{A}$ liegt mit $\mu(D \setminus E) = 0$ und $f$ $\script{A}|_E$-messbar.
        \item \glqq$f=g \mu$-fast überall\grqq ist eine Äquivalenzrelation auf der Menge aller Funktionen
        \item Sei $D \in \script{A}, f: D \to \bar{\mathbb{R}} \mu$-messbar. Dann ex. eine $\script{A}$-messbare Funktion $g: X \to \bar{\mathbb{R}}$ mit $f=g$ auf $D$, z.B.: $g = \begin{cases}
          f & \text{, auf } D\\
          0 & \text{, auf } X \setminus D 
        \end{cases}$\\
        Somit übertragen sich die Sätze I.16 und I.17 auf $\mu$-messbare Funktionen.
      \end{enumerate}
    \end{remark}

    \newpage
    \sidenote{Vorlesung 4}{13.11.20}
    \begin{lemma}
      $(X, \script{A}, \mu)$ vollständiger Maßraum. $f$ $\mu$-messbar auf $X$. Dann ist auch jede Funktion $\tilde{f}$ mit $\tilde{f}=f$ $\mu$-fast überall $\mu$-messbar.
    \end{lemma}

    \begin{proof}
      Sei $f$ auf $D \in \script{A}$ definiert mit $\mu(X \setminus D) = 0$ und sei $\tilde{f}$ auf $\tilde{D} \subseteq X$ definiert.\\
      Vor. $\implies \exists$ Nullmenge $N$ mit $X \setminus N \subseteq \cap \tilde{D}$ und $\tilde{f}(x) = f(x) \ \forall x \in X \setminus N$\\
      $\implies X \setminus \tilde{D} \subseteq N$\\
      $\stackrel{\mu\text{-vollständig}}{\implies} X\setminus \tilde{D} \in \script{A} \implies \tilde{D} \in \script{A}$.\\ \\
      Weiter gilt: \\
      $\{x \in \tilde{D}| \tilde{f}(x) < s\} = \{x \in \tilde{D} \cap N |\ \tilde{f}(x) < s\} \cup \{x \in \tilde{D} \cap (X \setminus N)|\ \tilde{f}(x) < s\}$\\
      $ = \{x \in \tilde{D} \cap N |\ \tilde{f}(x) < s\} \cup \{x \in D \cap (X \setminus N) |\ f(x) < s\}$\\
      $ = \{x \in \tilde{D} \cap N |\ \tilde{f}(x) < s\} \cup \{x \in D|\ f(x) < s\} \setminus \{x \in D \cap N |\ f(x) < s\}$\\
      $=: A \cup B$\\
      Da $f$ $\mu$-messbar ist, folgt, dass $B \in \script{A}$\\
      $\mu$-vollständig $\implies A \in \script{A} \implies \{x \in \tilde{D} |\ \tilde{f}(x) < s\} \in \script{A} \ \forall s$\\ \\
      Weiter ist $\{x \in \tilde{D}|\ \tilde{f}(x) < s\} \subseteq \tilde{D} \implies \{x \in \tilde{D}|\ \tilde{f}(x) < s\} \in \script{A}|_{\tilde{D}}$\\
      $\stackrel{\text{Lemma I.14}}{\Leftrightarrow} \tilde{f}$ $\mu$-messbar
    \end{proof}

    \begin{theorem}
      $(X, \script{A}, \mu)$ vollständiger Maßraum und seien $f_k, k \in \mathbb{N}$, $\mu$-messbar. Falls $f_k$ punktweise $\mu$-fast überall gegen $f$ konvergiert, dann ist $f$ auch $\mu$-messbar.
    \end{theorem}

    \begin{proof}
      Sei $f_k$ auf $D_k \in \script{A}$ definiert. Dann sind alle $f_k$, $k \in \mathbb{N}$, auf $D := \bigcap\limits_{k \in \mathbb{N}} D_k$ definiert und $X \setminus D$ ist $\mu$-Nullmenge $E := \{x \in D |\ \lim\limits_{k \to \infty} f_k(x) \neq f(x)\}$ und betrachte
      \begin{center}
        $\tilde{f}_k(x) = \begin{cases}
          f_k(x) & ,\forall x \in D \setminus E\\
          0 & \text{, sonst}
        \end{cases}$, $\tilde{f}(x) = \begin{cases}
          f(x) & ,\forall x \in D \setminus E\\
          0 & \text{, sonst}
        \end{cases}$
      \end{center}
      Es gilt $\tilde{f} = \lim\limits_{k \to \infty} \tilde{f}_k \stackrel{\text{Satz I.16}}{\implies} \tilde{f}$ ist $\script{A}$-messbar\\
      Vor.: $(X\setminus D) \cup E$ ist $\mu$-Nullmenge $\stackrel{\text{Lemma I.14}}{\implies} f$ ist $\mu$-messbar.
    \end{proof}

    \newpage

    \begin{theorem}[Egorov]
      $(X, \script{A}, \mu)$ Maßraum, $D \in \script{A}$ Menge mit $\mu(D) < \infty$ und $f_n, f$ $\mu$-messbare, $\mu$-fast überall endliche Funktionen auf $D$ mit $f_n \to f$ $\mu$-fast überall. Dann existiert $\forall \epsilon > 0$ eine Menge $B \in \script{A}$ mit $B \subseteq D$ und
      \begin{enumerate}[label=(\roman*)]
        \item $\mu(D \setminus B) < \epsilon$
        \item $f_n \to f$ gleichmäßig auf $B$
      \end{enumerate}
    \end{theorem}

    \begin{proof}
      $E := \{x \in D |\ f_n(x), f(x) \text{ sind endlich und } f_n(x) \to f(x)\}$\\
      Vor. $\implies \exists$ $\mu$-Nullmenge $N$ mit $D\setminus E \subseteq N$\\
      O.B. $E = D$ (sonst erstetze $D$ durch $D \setminus N$)\\
      Sei $C_{i,j} := \bigcup\limits_{n=j}^{\infty} \{x \in D |\ |f_n(x) - f(x)| > 2^{-i}\}, \ i,j \in \mathbb{N}$\\
      Satz I.17 $\implies$ $C_{i,j} \in \script{A}$ und $C_{i,j+1} \subseteq C_{i,j} \ \forall i,j \in \mathbb{N}$\\
      $\mu(D) < \infty \stackrel{\text{Satz I.7}}{\implies} \lim\limits_{j \to \infty} \mu(C_{i,j}) = \mu(\bigcap\limits_{j \in \mathbb N} C_{i,j}) = 0$, denn $f_n \to f$\\
      Sei $\epsilon > 0$ gegeben\\
      $\implies \forall i \in \mathbb{N} \ \exists N(i) \in \mathbb{N}$ mit $\mu(C_{i,N(i)}) < \epsilon * 2^{-i}$\\
      Setze $B := D \setminus \bigcup\limits_{i \in \mathbb{N}} C_{i,N(i)} \in \script{A}$ und $\mu(D \setminus B) = \mu(\bigcup\limits_{i \in \mathbb{N}} C_{i,N(i)}) \stackrel{\text{Satz I.7}}{\leq} \sum\limits_{i \in \mathbb{N}} \mu(C_{i,N(i)}) < \epsilon$\\
      $\forall i\in \mathbb{N} \ \forall x \in B \ \forall n>N(i)$ gilt:\\
      $|f_n(x) - f(x)| \leq 2^{-i}$
      $\implies f_n \to f$ auf $B$
    \end{proof}

  \chapter{Äußere Maße}
    \begin{definition}
      Sei $X$ eine Menge. Eine Funktion $\mu: \script{P}(X) \to [0,\infty]$ mit $\mu(\emptyset)=0$ heißt \textbf{äußeres Maß} auf X, falls gilt:
      \begin{center}
        $A \subseteq \bigcup\limits_{i \in \mathbb{N}} A_i \implies \mu(A) \leq \sum\limits_{i \in \mathbb{N}} \mu(A_i)$
      \end{center}
    \end{definition}

    \begin{remark}
      \begin{enumerate}
        \item[]
        \item Die Begriffe $\sigma$-additiv, $\sigma$-subadditiv, $\sigma$-endlich, endlich, monoton sowie Nullmenge und $\mu$-fast überall werden wie für Maße definiert. (Man ersetze überall $\script{A}$ durch $\script{P}(X)$)
        \item Ein äußeres Maß ist monoton, $\sigma$-subadditiv und insbesondere endlich subadditiv\\
          (d.h. $A \subseteq \bigcup\limits_{i=1}^n A_i \implies \mu(A) \leq \sum\limits_{i = 1}^n \mu(A_i)$)
      \end{enumerate}
    \end{remark}

    \begin{definition}
      Sei $\mu$ äußeres Maß auf $X$. Die Menge $A \subseteq X$ heißt \textbf{$\bm{\mu}$-messbar}, falls $\forall S \subseteq X$ gilt:
      \begin{center}
        $\mu(S) \geq \mu(S \cap A) + \mu(S \setminus A)$.
      \end{center}
      Das System aller $\mu$-messbaren Mengen wird mit $\bm{\script{M}(\mu)}$ bezeichnet.
    \end{definition}

    \begin{remark}
      Da $S = (S \cap A) \cup (S \setminus A)$ folgt aus Def. II.1:
      \begin{center}
        $\mu(S) \leq \mu(S \cap A) + \mu(S \setminus A)$
      \end{center}
      d.h.: $A$ messbar $\Leftrightarrow \mu(S \cap A) + \mu(S \setminus A) \ \forall S \subseteq X$ 
    \end{remark}

    \begin{example}
      Jedes auf $\script{P}(X)$ definierte Maß ist ein äußeres Maß (Satz I.7), also sind das DiracMaß und das Zählmaß äußere Maße.
    \end{example}

    \newpage
    \begin{theorem}
      Sei $\script{Q}$ ein System von Teilmengen einer Menge $X$, welches die leere Menge enthält, und sei $\lambda: \script{Q} \to [0,\infty]$ eine Mengenfunktion auf $\script{Q}$ mit $\lambda(\emptyset)=0$. Definiere die Mengenfunktion $\mu(E):= \inf\{\sum\limits_{i \in \mathbb{N}} \lambda(P_i) |\ P_i \in \script{Q}, E \subseteq \bigcup\limits_{i \in \mathbb{N}} P_i\}$.\\
      Dann ist $\mu$ ein äußeres Maß. \hfill ($\inf \emptyset = \infty$)
    \end{theorem}

    \begin{proof}
      Mit $\emptyset \subseteq \emptyset \in \script{Q}$ folgt $\mu(\emptyset) = 0$.\\
      Sei $E \subseteq \bigcup\limits_{i \in \mathbb{N}} E_i$ mit $E, E_i \subseteq X$ und $\mu(E_i) < \infty$.\\ \\
      \underline{z.z.:} $\mu(E) \leq \sum\limits_{i \in \mathbb{N}} \mu(E_i)$\\
      Wähle Überdeckungen $E_i \subseteq \bigcup\limits_{j \in \mathbb{N}} P_{i,j}$ mit $P_{i,j} \in \script{Q}$, so dass zu $\epsilon > 0$ gegeben gilt:
      \begin{center}
        $\sum\limits_{j \in \mathbb{N}} \lambda(P_{i,j}) < \mu(E_i) + 2^{-i} * \epsilon \ , \forall i \in \mathbb{N}$
      \end{center}
      $\implies E \subseteq \bigcup\limits_{i,j \in \mathbb{N}} P_{i,j}$ und damit $\mu(E) \leq \sum\limits_{i,j \in \mathbb{N}} \lambda(P_{i,j}) \leq \sum\limits_{i \in \mathbb{N}} (\mu(E_i) + 2^{-i} * \epsilon) = \sum\limits_{i \in \mathbb{N}} \mu(E_i) + \epsilon$\\
      Mit $\epsilon > 0$ folgt $\mu(E) \leq \sum\limits_{i \in \mathbb{N}} \mu(E_i)$
    \end{proof}

    \begin{theorem}
      Sei $\mu: \script{P}(X) \to [0, \infty]$ äußeres Maß auf X. Für M $\subseteq X$ gegeben erhält man durch $\mu \llcorner M: \script{P}(X) \to [0, \infty], \mu \llcorner M(A) := \mu(A \cap M)$ ein äußeres Maß $\mu \llcorner M$ auf $X$, welches wir \textbf{Einschränkung} von $\mu$ auf M nennen.\\
      Es gilt:
      \begin{center}
        $A$ $\mu$-messbar $\implies$ $A$ $\mu \llcorner M$-messbar
      \end{center} 
    \end{theorem}

    \begin{proof}
      Aus der Definition folgt sofort, dass $\mu \llcorner M$ ein äußeres Maß ist. Weiter gilt für $A \subseteq X$ $\mu$-messbar und $S \subseteq X$ beliebig:
      \begin{align*}
        \mu \llcorner M(S) &= \mu (S \cap M)\\
          &\geq \mu((S \cap M) \cap A) + \mu((S \cap M) \setminus A)\\
          &= \mu ((S \cap A) \cap M) + \mu ((S \setminus A) \cap M)\\
          &= \mu \llcorner M (S \cap A) + \mu \llcorner M (S \setminus A)
      \end{align*}
      $\implies$ Behauptung
    \end{proof}

    \newpage
    \begin{theorem}
      $\mu$ äußeres Maß auf $X$. Dann gilt:
      \begin{align*}
        N \ \mu\text{-Nullmenge} &\implies N \ \mu\text{-messbar}\\
        N_k, k \in \mathbb{N}, \mu\text{-Nullmengen} &\implies \bigcup\limits_{k \in \mathbb{N}} N_k \ \mu\text{-Nullmenge}
      \end{align*}
    \end{theorem}

    \begin{proof}
      Sei $\mu(N) = 0$. Für $S \subseteq X$ folgt aus Monotonie:\\
      $\mu(S\cap N) \leq \mu(N) = 0$, $\mu(S) \geq \mu(S \setminus N) = \mu(S \cap N) + \mu(S \setminus N) \implies N$ $\mu$-messbar\\
      Zweite Behauptung folgt aus $\sigma$-Subadditivität.
    \end{proof}

    \begin{remark}
      $\script{M}(\mu)$ enthält alle Nullmengen $N \subseteq X$ und damit auch deren Komplemente\\
      (siehe Satz II.7). Es kann sein, dass keine anderen Mengen $\mu$-messbar sind.
    \end{remark}

    \begin{example}
      Auf $X$ bel. definiere: $\beta(A) = \begin{cases}
        0 & , A = \emptyset\\
        1 & , \text{ sonst}
      \end{cases}$
      $\beta$ ist äußeres Maß.\\
      Es sind nur $\emptyset$ und $X$ $\beta$-messbar, denn für $X=S$ folgt aus der Annahme,\\
      dass $A$ $\beta$-messbar ist: $1 \geq \beta(A) + \beta(X \setminus A)$
    \end{example}

    \sidenote{Vorlesung 5}{16.11.20}

    \begin{lemma}
      Seien $A_i \in \script{M}(\mu)$, $i=1,...,k$, paarweiße disjunkt und $\mu$ äußeres Maß. Dann gilt $\forall S \subseteq X:$
      \begin{center}
        $\mu(S \cap \bigcup\limits_{i=1}^k A_i) = \sum\limits_{i=1}^k \mu(S \cap A_i)$
      \end{center}
    \end{lemma}

    \begin{proof}
      \underline{$k=1$:} trivial\\
      \underline{$k \geq 2$:} $A_k$ $\mu$-messbar\\
      \begin{align*}
        \mu(S \cap \bigcup\limits_{i=1}^k A_i) 
        &= \mu((S \cap \bigcup\limits_{i=1}^k A_i) \cap A_k) + \mu((S \cap \bigcup\limits_{i=1}^k A_i) \setminus A_k)\\
        &=\mu(S \cap A_k) + \mu(S \cap \bigcup\limits_{i=1}^k A_k)\\
        &\stackrel{\text{IV}}{=} \sum\limits_{i=1}^k \mu(S \cap A_i) 
      \end{align*}
    \end{proof}

    \begin{theorem}
      Sei $\mu: \script{P}(X) \to [0,\infty]$ ein äußeres Maß. Dann ist $\script{M}(\mu)$ eine $\sigma$-Algebra und $\mu$ ist ein vollständiges Maß auf $\script{M}(\mu)$.
    \end{theorem}

    \begin{proof}
      Notation: Schreibe $\script{M}$ statt $\script{M}(\mu)$\\
      Es gilt:
      \begin{itemize}
        \item $x \in \script{M}$, denn: $\forall S \subseteq X$ ist:\\
              $\mu(S \cap X) + \mu(S \setminus X) = \mu(S) + \mu(\emptyset) = \mu(S)$
        \item Sei $A \in \script{M} \implies X \setminus A \in \script{M}$, denn $\forall S \subset X$ gilt:\\
              $\mu(S \cap (X \setminus A)) + \mu(S \setminus (X \setminus A)) = \mu(S \setminus A) + \mu(S \cap A) = \mu(S)$
      \end{itemize}
      Als nächstes zeigen wir:\\
      $A,B \in \script{M} \implies A \cap B \in \script{M} \ \forall S \subseteq X$ gilt:
      \begin{align*}
        \mu(S) 
        &= \mu(S \cap A) + \mu(S \setminus A)\\
        \mu(S \cap A) 
        &= \mu(S \cap A \cap B) + \mu((S \cap A) \setminus B)\\
        \mu(S \setminus (A \cap B)) 
        &= \mu((S \setminus (A \cap B)) \cap A) + \mu((S \setminus (A \cap B)) \setminus A)\\
        &= \mu((S \cap A) \setminus B) + \mu(S \setminus A)\\
      \end{align*}
      $\implies \mu(S) = \mu(S \cap (A \cap B)) + \mu(S \setminus (A \cap B))$\\
      $\implies A \cup B \in \script{M}$, denn:\\
      $A \cup B = X \setminus ((X \setminus A) \cap (X \setminus B))$\\ \\
      Per Induktion:\\
      $\script{M}$ ist abgeschlossen unter endlichen Durchschnitten und Vereinigungen.\\ \\
      \underline{Jetzt:} $\mu$ ist $\sigma$-additiv auf $\script{M}$.\\
      Seien $A_j, j \in \mathbb{N},$ paarweiße disjunkt mit $A_j \in \script{M} \ \forall j \in \mathbb{N}$\\
      Wähle $S = A_1 \cup A_2$ und benutze $A_1 \in \script{M}$\\
      $\implies \mu(S) = \mu(A_1 \cup A_2) = \mu(A_1) + \mu(A_2) \ \ (= \mu(S \cap A_1) + \mu(S \setminus A_1))$\\ \\
      Induktion: Dasselbe gilt für endliche disjunkte Vereinigungen. \begin{align*}
        \sum\limits_{j \in \mathbb{N}} \mu(A_j)
        &= \lim\limits_{k \to \infty} \sum\limits_{j = 1}^k \mu(A_j)
        = \lim\limits_{k \to \infty} \mu(\bigcup\limits_{j=1}^k A_j)\\
        &\leq \mu(\bigcup\limits_{j \in \mathbb{N}} A_j)
        \stackrel{\sigma \text{-Subadd.}}{\leq} \sum\limits_{j=1}^k \mu(A_j)
      \end{align*}
      $\implies \mu(\bigcup\limits_{j \in \mathbb{N}} A_j) = \sum\limits_{j \in \mathbb{N}} \mu(A_j) \implies$ Behauptung\\ \\
      Als letztes: $\script{M}$ ist abgeschlossen unter abzählbaren Vereinigungen\\
      Seien $A_j \in \script{M}, j \in \mathbb{N}$. O.B. seien $A_j$ paarweise disjunkt, sonst betrachte \\
      $\tilde{A_i} := A_i \setminus (A_1 \cup ... \cup A_{i-1})$\\
      Für $S \subseteq X$ folgt mit $\bigcup\limits_{i=1}^k A_i \in \script{M}$:
      \begin{align*}
        \mu(S) &= \mu(S \cap \bigcup\limits_{i=1}^k A_i) + \mu(S \setminus \bigcup\limits_{i=1}^k A_i)\\
        &\stackrel{\text{Lemma II.6}}{\geq} \sum\limits_{i=1}^k \mu(S \cap A_i) + \mu(S \setminus \bigcup\limits_{i \in \mathbb{N}} A_i) \ \ \forall k \in \mathbb{N}
      \end{align*}
      Lasse $k \to \infty$\\
      \begin{align*}
        \implies \mu(S) &\geq \sum\limits_{i \in \mathbb{N}} \mu(S \cap A_i) + \mu(S \setminus \bigcup\limits_{i \in \mathbb{N}} A_i)\\
        &\stackrel{\sigma\text{-Subadd.}}{\geq} \mu(\bigcup\limits_{i \in \mathbb{N}} (S \cap A_i)) + \mu(S \setminus \bigcup\limits_{i \in \mathbb{N}} A_i)\\
        &= \mu(S \cap (\bigcup\limits_{i \in \mathbb{N}} A_i)) + \mu(S \setminus \bigcup\limits_{i \in \mathbb{N}} A_i)\\
        &\implies \bigcup\limits_{i \in \mathbb{N}} A_i \in \script{M}
      \end{align*}
      Vollständigkeit von $\mu$: siehe Lemma II.5
    \end{proof}

    \begin{lemma}
      $\mu$ äußeres Maß, $A_i \in \script{M}(\mu), i \in \mathbb{N}$.\\
      Dann gelten:
      \begin{enumerate}[label=\roman*)]
        \item Aus $A_1 \subseteq ... \subseteq A_i \subseteq A_{i+1} \subseteq ...$ folgt $\mu(\bigcup\limits_{i \in \mathbb{N}} A_i) = \lim\limits_{i \to \infty} \mu(A_i)$
        \item Aus $A_1 \supseteq ... \supseteq A_i \supseteq A_{i+1} \supseteq ...$ mit $\mu(A_1) < \infty$ folgt $\mu(\bigcap\limits_{i \in \mathbb{N}} A_i) = \lim\limits_{i \to \infty} \mu(A_i)$
      \end{enumerate} 
    \end{lemma}
    
    \begin{proof}
      Folgt aus Satz I.7 und Satz II.7
    \end{proof}

    \begin{definition}
      Ein Mengensystem $\script{A} \subseteq \script{P}(X)$ heißt $\bm{\bigcup}$\textbf{-stabil} (bzw. $\bm{\bigcap}$\textbf{-stabil}, $\bm{\setminus}$\textbf{-stabil}), wenn $A \cup B \in \script{A}$ (bzw. $A \cap B \in \script{A}$, $A \setminus B \in \script{A}$) $\forall A,B \in \script{A}$ gilt.
    \end{definition}

    \begin{remark}
      $\bigcup$-stabil impliziert Stabilität bzgl. endlicher Vereinigung. Ebenso $\bigcap$-stabil.
    \end{remark}

    \begin{definition}
      Ein Mengensystem $\script{R}\subset\script{P}(X)$ heißt \textbf{Ring} über $X$, falls:
      \begin{enumerate}[label=\roman*)]
        \item $\emptyset \in \script{R}$
        \item $A,B \in \script{R} \implies A \setminus B \in \script{R}$
        \item $A,B \in \script{R} \implies A \cup B \in \script{R}$
      \end{enumerate}
      
      $\script{R}$ heißt \textbf{Algebra}, falls zusätzlich $X \in \script{R}$.
    \end{definition}

    \begin{example}
      \begin{enumerate}[label=\roman*)]
        \item[]
        \item Für $A \subset X$ ist $\{\emptyset, A\}$ ein Ring, aber für $A \neq X$ keine Algebra.
        \item System aller endlichen Teilmengen einer bel. Menge ist ein Ring.
        \item Ebenso System aller höchstens abzählbaren Teilmengen. 
      \end{enumerate}
    \end{example}

    \begin{remark}
      Für $A,B \in \script{R}$ gilt: $A \cap B = A \setminus (A \setminus B) \in \script{R}$\\
      Ringe sind $\bigcup$-stabil, $\bigcap$-stabil, $\setminus$-stabil
    \end{remark}

    \begin{definition}[Im Aufschrieb II.10]
      Sei $\script{R} \subseteq \script{P}(X)$ Ring. Eine Funktion $\lambda: \script{R} \to [0, \infty]$ heißt \textbf{Prämaß} auf $\script{R}$, falls:
      \begin{enumerate}[label=\roman*)]
        \item $\lambda(\emptyset) = 0$
        \item Für $A_i \in \script{R}, i \in \mathbb{N}$, paarweiße disjunkt mit $\bigcup\limits_{i \in \mathbb{N}} A_i \in \script{R}$ gilt:\\
        $\lambda(\bigcup\limits_{i \in \mathbb{N}} A_i) = \sum\limits_{i \in \mathbb{N}} \lambda(A_i) $
      \end{enumerate}
    \end{definition}

    \begin{remark}
      $\sigma$-subadditiv, subadditiv, $\sigma$-endlich, endlich, monoton, Nullmenge und fast-überall werden wie für Maße definiert. 
    \end{remark}

    \begin{example}
      \begin{enumerate}[label=\roman*)]
        \item[]
        \item $\script{R}$ Ring über $X$. $\lambda(A) = \begin{cases}
                0 & H = \emptyset\\
                \infty & \text{sonst}
              \end{cases}$
        \item $\script{R}$ sei Ring der endlichen Teilmengen einer beliebigen Menge $X$ und $\lambda = card|_\script{R}$ ist Prämaß
        \item Alle Maße sind Prämaße. Inbesondere äußere Maße eingeschränkt auf die messbaren Mengen.
      \end{enumerate}
    \end{example}

    \begin{definition}[Im Aufschrieb II.11]
      $\lambda$ Prämaß auf Ring $\script{R} \subseteq \script{P}(X)$. Ein äußeres Maß $\mu$ auf $X$ (bzw. ein Maß auf $\script{A}$) heißt \textbf{Fortsetzung} von $\lambda$, falls gilt:
      \begin{enumerate}[label=\roman*)]
        \item $\mu|_\script{R} = \lambda$, d.h. $\mu(A) = \lambda(A) \ \forall A \in \script{R}$
        \item $\script{R} \subseteq \script{M}(\mu)$ (bzw. $\script{R} \subset \script{A}$), d.h. alle $A \in \script{R}$ sind $\mu$-messbar
      \end{enumerate}
    \end{definition}

    \begin{theorem}[Caratheodory-Fortsetzung | Im Aufschrieb II.12]
      $\lambda: \script{R} \to [0, \infty]$ Prämaß auf Ring $\script{R} \subseteq \script{P}(X)$. Sei $\mu: \script{P}(X) \to [0, \infty]$ das in Satz II.3 aus $\script{R}$ konstruierte äußere Maß, d.h. $\forall E \subseteq X:$
      \begin{align*}
        \mu(E) := inf\{\sum\limits_{i \in \mathbb{N}} \lambda(A_i) \ | \ A_i \in \script{R}, E \subseteq \bigcup\limits_{i \in \mathbb{N}} A_i\}
      \end{align*}
      Dann ist $\mu$ eine Fortsetzung von $\lambda$.\\
      $\mu$ heißt \textbf{induziertes äußeres Maß} oder \textbf{Caratheodory-Fortsetzung} von $\lambda$.
    \end{theorem}

    \begin{proof}
      \begin{enumerate}[label=\roman*)]
        \item[]
        \item $\mu(A) = \lambda(A) \ \forall A \in \script{R}$\\
              Wir haben $\mu(A) \leq \lambda(A)$ aus Def. mit $A_1 = A, A_2 = ... = \emptyset$\\
              Für $\lambda(A) \leq \mu(A)$ reicht es zz, dass:\\
              $A \subseteq \bigcup\limits_{i \in \mathbb{N}} A_i$ mit $A_i \in \script{R} \implies \lambda(A) \leq \sum\limits_{i \in \mathbb{N}} \lambda(A_i)$\\
              Betrachte paarweise disjunkte Mengen $B_i = (A_i \setminus \bigcup\limits_{j=1}^{i-1} A_j) \cap A \in \script{R}$\\
              $\implies \lambda(A) = \lambda(\bigcup\limits_{i \in \mathbb{N}} B_i) = \sum\limits_{i \in \mathbb{N}} \lambda(B_i) \leq \sum\limits_{i \in \mathbb{N}} \lambda(A_i)$
        \item Jedes $A \in \script{R}$ ist $\mu$-messbar.\\
              Sei $A \in \script{R}, S \subseteq X$ bel. mit $\mu(S) < \infty$. Zu $\epsilon > 0$ wähle $A_i \in \script{R}$, sodass $S \subseteq \bigcup\limits_{i \in \mathbb{N}} (A_i \cap A)$ und $S \setminus A \subseteq \bigcup\limits_{i \in \mathbb{N}} (A_i \setminus A)$
              \begin{align*}
                \implies \mu(S \cap A) + \mu(S \setminus A) 
                &\leq \sum\limits_{i \in \mathbb{N}} \lambda(A_i \cap A) + \sum\limits_{i \in \mathbb{N}} \lambda(A_i \setminus A)\\
                &= \sum\limits_{i \in \mathbb{N}} \lambda(A_i) \leq \mu(S) + \epsilon
              \end{align*}
              Lasse $s \downarrow 0 \implies A \in \script{M}(\mu)$\\
              Für $\mu(S) = \infty$ ist das trivial.
      \end{enumerate}
    \end{proof}

    \newpage

    \begin{lemma}[Im Aufschrieb II.13]
      $\mu$ sei Caratheodory-Fortsetzung des Prämaßes $\lambda: \script{R} \to [0, \infty]$ auf dem Ring $\script{R}$ über $X$. Sei $\tilde{\mu}$ ein Maß auf $\sigma(\script{R})$ mit $\tilde{\mu} = \mu$ auf $\script{R}$, dann gilt $\forall E \in \sigma(\script{R})$:\\
      $\tilde{\mu}(E) \leq \mu(E)$
    \end{lemma}

    \begin{proof}
      $\forall E \in \sigma(\script{R}): E \subseteq \bigcup\limits_{i \in \mathbb{N}} P_i$ mit $P_i \in \script{R}$\\
      $\implies \tilde{\mu}(E) \leq \sum\limits_{i \in \mathbb{N}} \tilde{\mu}(P_i) = \sum\limits_{i \in \mathbb{N}} \lambda(P_i)$\\
      Bilde Infimum über alle solche Überdeckungen\\
      $\implies \tilde{\mu}(E) \leq \mu(E)$ 
    \end{proof}

    \sidenote{Vorlesung 6}{20.11.20}

    \begin{theorem}[Im Aufschrieb II.14]
      Sei $\lambda: \script{R} \to [0, \infty]$ Prämaß auf Ring $\script{R}\subseteq \script{P}(X)$. Dann ex. ein Maß $\mu$ auf $\sigma(\script{R})$ mit $\mu=\lambda$ auf $\script{R}$. Diese Fortsetzung ist eindeutig, falls $\lambda$ $\sigma$-endlich ist.
    \end{theorem}

    \begin{proof}
      siehe Aufschrieb
    \end{proof}

    \begin{theorem}[Regularität der Caratheodory-Fortsetzung | i.A. II.15]
      Sei $\mu$ Caratheodory-Fortsetzung des Prämaßes $\lambda: \script{R} \to [0,\infty]$ auf Ring $\script{R}$ über $X$. Dann ex. $\forall D \subseteq X$ ein $E \in \sigma(\script{R})$ mit $E \supseteq D$ und $\mu(E) = \mu(D)$.\\
      ($\mu$ ist \glqq reguläres \grqq äußeres Maß)
    \end{theorem}

    \begin{proof}
      siehe Aufschrieb
    \end{proof}

    \begin{theorem}[i.A. II.16]
      Sei $\lambda$ ein $\sigma$-endliches Prämaß auf Ring $\script{R}$ über $X$ und sei $\mu: \script{P}(X) \to [0,\infty]$ die Caratheodory-Fortsetzung von $\lambda$. Dann ist $\mu|_{\script{M}(\mu)}$ die Vervollständigung von $\mu|_{\sigma(\script{R})}$ und $\script{M}(\mu)$ ist die vervollständigte $\sigma$-Algebra von $\overline{\sigma(\mathbb{R})}_{\mu|_{\sigma(\mathbb{R})}}$.\\
      D.h. $\overline{\sigma(\mathbb{R})}_{\mu|_{\sigma(\mathbb{R})}} = \script{M}(\mu)$. Insbesondere ex. genau eine Fortsetzung von $\lambda: \script{R} \to [0, \infty]$ zu einem vollständigen Maß auf $\script{M}(\mu)$.
    \end{theorem}

    \begin{proof}
      siehe Aufschrieb
    \end{proof}

    \newpage

    \begin{lemma}[i.A. II.17]
      $\lambda: \script{R} \to [0, \infty]$ $\sigma$-endliches Prämaß auf Ring $\script{R} \subseteq \script{P}(X)$ mit Caratheodory-Fortsetzung $\mu$. $D \subseteq X$ ist genau dann $\mu$-messbar, wenn eine der folgenden Bedingungen gilt:
      \begin{enumerate}[label=\roman*)]
        \item $\exists \ E \in \sigma(\script{R})$ mit $E \supseteq D$ und $\mu(E \setminus D) = 0$
        \item $\exists \ C \in \sigma(\script{R})$ mit $C \subseteq D$ und $\mu(D \setminus C) = 0$
      \end{enumerate}
    \end{lemma}

    \begin{definition}
      Ein Mengensystem $\script{Q} \subseteq \script{P}(X)$ heißt \textbf{Halbring} über $X$, falls:
      \begin{enumerate}[label=\roman*)]
        \item $\emptyset \in \script{Q}$
        \item $P, Q \in \script{Q} \implies P \cap Q \in \script{Q}$
        \item $P, Q \in \script{Q} \implies P \setminus Q = \bigcup\limits_{i=1}^k P_i$ mit endlich vielen paarweise disjunkten $P_i \in \script{Q}$
      \end{enumerate}
    \end{definition}

    \begin{example}
      $X$ beliebige Menge. $\script{Q} := \{\emptyset\} \cup \{\{a\} \ | \ a \in X\}$
    \end{example}

    \begin{remark}
      $I \subseteq \mathbb{R}$ heißt \textbf{Intervall}, wenn es $a,b \in \mathbb{R}$ mit $a \leq b$ gibt, sodass: $(a,b) \subseteq I \subseteq [a,b]$. Das System aller Intervalle bezeichnen wir mit $\script{I}$.\\
      Ein achsenparalleler n-dim. \textbf{Quader} (kurz: Quader) ist Produkt $Q = I_1 \times ... \times I_n \subseteq \mathbb{R}^n$ von Intervallen. Das System aller Quader wird mit $\script{Q}^n$ bezeichnet.
    \end{remark}

    \begin{theorem}[i.A. II.19]
      $\script{I}$ ist ein Halbring.
    \end{theorem}

    \begin{proof}
      siehe Aufschrieb
    \end{proof}

    \begin{theorem}[i.A. II.20]
      Für $i = 1, ..., n$ sei $\script{Q}_i$ Halbring über $X_i$. Dann ist $\script{Q}:=\{P_1 \times ... \times P_n \ | \ P_i \in \script{Q}_i\}$ ein Halbring über $X_1 \times ... \times X_n$.
    \end{theorem}

    \begin{proof}
      siehe Aufschrieb
    \end{proof}

    \begin{theorem}[i.A. II.21]
      $\script{Q}^n$ ist ein Halbdring.
    \end{theorem}
\end{document}