\chapter{Lebesgue-Integral}
  \begin{definition}
    $X$ Menge, $\mu$ äußeres Maß. Eine funktion $\zeta: X \to \mathbb{R}$ heißt $\bm{\mu}$\textbf{-Treppenfunktion}, wenn sie $\mu$-messbar ist und nur eindlich viele Funktionswerte annimmt.\\
    Die Menge $\script{T}(\mu)$ der $\mu$-Treppenfunktionen ist ein $\mathbb{R}$-Vektorraum. Wir setzen
    \begin{align*}
      \script{T}^+(\mu)=\{\zeta \in \script{T}(\mu) \ | \ \zeta \geq 0\}
    \end{align*}
  \end{definition}

  \begin{example}
    $E \subseteq X, \psi_E: X \to \mathbb{R}, \psi_E(x) = \begin{cases}
      1 & ,x \in E\\
      0 & ,\text{ sonst}
    \end{cases}$  Es ist: $\psi_E$ $\mu$-Treppenfunktion $\Leftrightarrow E \in \script{M}(\mu)$\\
    Sei $\zeta \geq 0, \zeta = \sum\limits_{i=1}^k s_i \psi_{A_i}$ mit $A_i$ messbar und $s_i \geq 0$ und die $A_i$ sind paarweise disjunkt. So eine Darstellung heißt \textbf{einfach}.\\
    Wir setzen:
    \begin{align*}
      (\star) \ I(\zeta) := \sum\limits_{i=1}^k s_i \mu(A_i)
    \end{align*} 
    Für $\zeta=0$ folgt $I(\zeta) = 0 \cdot \mu(X) = 0$\\
    Jedes $\zeta \in \script{T}^+(\mu)$ besitzt eine einfache Darstellung, z.B. können wir für $s_i$ die endlich vielen Funktionswerte wählen und $A_i = \{\zeta = s_i\}$
  \end{example}

  \begin{lemma}
    Das Integral $I: \script{T}^+(\mu) \to [0,\infty]$ ist durch $(\star)$ wohldefiniert. Für $\zeta, \phi \in \script{T}^+(\mu)$ und $\alpha, \beta \in [0, \infty)$ gilt:
    \begin{enumerate}[label=\roman*)]
      \item $I(\alpha \zeta + \beta \psi) = \alpha I(\zeta) + \beta I(\psi)$
      \item $\zeta \leq \psi \implies I(\zeta) \leq I(\psi)$ 
    \end{enumerate}
  \end{lemma}

  \begin{proof}
    siehe Aufschrieb
  \end{proof}

  \begin{remark}
    Für $A_i$ messbar und $s_i \geq 0$ folgt aus i) auch für $A_i$ nicht disjunk:
    \begin{align*}
      I(\zeta) = \sum\limits_{i=1}^k s_i \mu(A_i) \ \ \text{ für } \zeta = \sum\limits_{i=1}^k s_i \psi_{A_i}
    \end{align*}
  \end{remark}

  \begin{definition}[Lebesgue-Integral]
    Für $f: X \to [0,\infty]$ $\mu$-messbar, setze
    \begin{align*}
      \int f d\mu = sup\{I(\zeta) \ | \ \zeta \in \script{T}^+(\mu), \zeta \leq f\}
    \end{align*}
    $\zeta$ heißt \textbf{Unterfunktion} von f.\\
    Ist $f: X \to [-\infty, \infty]$ $\mu$-messbar und sind die Integrale von $f^{\pm}$ nicht beide unendlich, so setzen wir
    \begin{align*}
      \int f d\mu = \int f^+ d\mu - \int f^- d\mu \ \ \in [-\infty, \infty] 
    \end{align*}
  \end{definition}

  \begin{remark}
    Für $f \geq 0$ sind beide Schritte kompatibel, denn dann gilt $f = f^+$ und $f^- = 0$
  \end{remark}

  \begin{lemma}
    Für $f \in \script{T}^+(\mu)$ gilt: $\int f d\mu = I(f)$
  \end{lemma}

  \begin{proof}
    siehe Aufschrieb
  \end{proof}

  \begin{example}
    $\chi_{\mathbb{Q}}$ ist eine $\lambda^1$-Treppenfunktion und es gilt:\\
    $\int \chi_{\mathbb{Q}} d\lambda^1 = I(\chi_{\mathbb{Q}}) = 0 \cdot \lambda^1(\mathbb{R} \setminus \mathbb{Q}) + 1 \cdot \lambda^1(\mathbb{Q}) = 0 + 1 \cdot 0 = 0$
  \end{example}

  \begin{definition}
    $f:X \to \bar{\mathbb{R}}$ heißt \textbf{integrierbar} bzgl. $\mu$, wenn sie $\mu$-messbar ist und wenn gilt:
    \begin{align*}
      \int f d\mu \in \mathbb{R} \Leftrightarrow \int f^+ d\mu + \int f^- d\mu < \infty
    \end{align*}
  \end{definition}

  \begin{example}
    $\mu = card, X = \mathbb{N}_0$\\
    z.z.: $f: \mathbb{N}_0 \to \mathbb{R}$ ist bzgl. $card$ auf $\mathbb{N}_0$ integrierbar $\implies \sum\limits_{k \in \mathbb{N}} f(k)$ absolut konvergent\\
    Dann gilt: $\int f d card = \sum\limits_{k \in \mathbb{N}} f(k)$\\
    Beweis siehe Aufschrieb
  \end{example}

  \begin{theorem}
    $f,g:X \to \bar{\mathbb{R}}$ $\mu$-messbar. Ist $f \leq g$ $\mu$-fast überall und $\int f^- d\mu < \infty$, so existieren beide Integrale und es ist: $\int f d\mu \leq \int g d\mu$\\
    \glqq$\geq$\grqq gilt entsprechend wenn $f^+ d\mu < \infty$
  \end{theorem}

  \begin{proof}
    siehe Aufschrieb
  \end{proof}

  \begin{remark}
    $f,g: X \to \bar{\mathbb{R}}$, $f$ $\mu$-messbar und $g = f$ $\mu$-fast überall $\stackrel{\text{Kapitel II}}{\implies} g$ $\mu$-messbar\\
    Satz IV.6 $\implies \int g^{\pm} d\mu = \int f^{\pm} d\mu \implies \int f d\mu = \int g \ d\mu$
  \end{remark}

  \sidenote{Vorlesung 12}{11.11.2020}

  \begin{remark}
    Einschub: zum Beweis von Satz III.7
    siehe Aufschrieb
  \end{remark}

  \begin{lemma}[Tschebyscheff-Ungleichung]
    Für $f:X \to [0, \infty]$ $\mu$-messbar mit $\int f d\mu < \infty$ gilt:
    \begin{align*}
      \mu(\{f\geq s\}) \leq \begin{cases}
        \dfrac{1}{s} \cdot \int f d\mu & \text{ für } s \in (0, \infty)\\
        0 & \text{ für } s = \infty
      \end{cases}
    \end{align*}
  \end{lemma}

  \begin{proof}
    siehe Aufschrieb
  \end{proof}

  \begin{lemma}
    Sei $f: X \to \bar{\mathbb{R}}$ $\mu$-messbar.
    \begin{enumerate}[label=\roman*)]
      \item ist $\int f d\mu < \infty \implies \{f = \infty\}$ ist $\mu$-Nulllmenge
      \item ist $f \geq 0$ und $\int f d\mu = 0 \implies \{f > 0\}$ ist $\mu$-Nullmenge
    \end{enumerate}
  \end{lemma}

  \begin{proof}
    siehe Aufschrieb
  \end{proof}

  \begin{theorem}
    Zu $f: X \to [0,\infty]$ $\mu$-messbar gibt es eine Folge $f_k \in \script{T}^+(\mu)$ mit $f_0 \leq f_1 \leq ...$ und $\lim\limits_{k \to \infty} f_k(x) = f(x) \ \forall x \in X$.
  \end{theorem}

  \begin{proof}
    siehe Aufschrieb
  \end{proof}

  \begin{theorem}[Monotonie Konvergenz / Beppo-Levi]
    Seien $f_k:X \to [0,\infty]$ $\mu$-messbar mit $f_1 \leq f_2 \leq ...$ und $f: X \to [0, \infty]$ mit $f(x) := \lim\limits_{k \to \infty} f_k(x)$. Dann gilt:
    \begin{align*}
      \int f d\mu = \lim\limits_{k \to \infty} \int f_k \ d\mu
    \end{align*}
  \end{theorem}

  \begin{proof}
    siehe Aufschrieb
  \end{proof}

  \begin{theorem}
    $f,g: X \to \bar{\mathbb{R}}$ integrierbar bzgl. $\mu$, so ist auch $\alpha f + \beta g$ integrierbar $\forall \alpha, \beta \in \mathbb{R}$ und es gilt:
    \begin{align*}
      \int (\alpha f + \beta g) \ d\mu = \alpha \int f d\mu + \beta \int g d\mu
    \end{align*}
  \end{theorem}

  \begin{proof}
    siehe Aufschrieb
  \end{proof}

  \begin{definition}
    Sei $\mu$ ein äußeres Maß auf $X$ und $E \subseteq X$ sei $\mu$-messbar. Dann setzen wir, falls das rechte Integral existiert
    \begin{align*}
      \int\limits_E f d\mu = \int f \chi_E d\mu
    \end{align*}
    $f$ heißt \textbf{auf $\bm{E}$ integrierbar}, wenn $f \chi_E$ integrierbar ist.
  \end{definition}

  \begin{remark}
    Wegen $(f \chi_E)^{\pm} = f^{\pm} \chi_E \leq f^{\pm}$ existiert das Integral von $f$ über $E$ auf jeden Fall dann, wenn $\in f d\mu$ existiert. (Speziell für $f \geq 0$)
  \end{remark}

  \begin{example}
    $\alpha \in \mathbb{R}, \ f:\mathbb{R}^n \to \mathbb{R}, \ f(x) = ||x||^{-\alpha}$\\
    Beh: 
    \begin{align*}
      \int\limits_{\mathbb{R}^n \setminus B_1(0)} f d\lambda^n < \infty &\Leftrightarrow \alpha > n\\
      \int\limits_{B_1(0)} f d\lambda^n < \infty &\Leftrightarrow \alpha < n
    \end{align*}
    Beweis siehe Aufschrieb
  \end{example}

  \sidenote{Vorlesung 13}{14.12.20}
  \begin{theorem}
    Sei $f: X \to \bar{\mathbb{R}}$ $\mu$-messbar. Dann gelten:
    \begin{enumerate}[label=\roman*)]
      \item $f$ integrierbar $\Leftrightarrow |f|$ integrierbar
      \item Es gilt: $|\int f d\mu| \leq \int |f| d\mu$, falls das Integral von $f$ existiert
      \item Ist $g: X \to [0, \infty]$ $\mu$-messbar mit $|f| \leq g$ $\mu$-fast überall und $\int g d\mu < \infty$, so ist $f$ integrierbar 
    \end{enumerate}
  \end{theorem}

  \begin{proof}
    siehe Aufschrieb
  \end{proof}

  \begin{example}
    $f: \mathbb{R}^n \to \bar{\mathbb{R}}$ $\lambda^n$-messbar und es gelte für ein $C \in [0, \infty]$:\\
    $|f(x)| \leq C ||x||^{-\alpha}$ fast überall in $B_{\epsilon}(0)$ mit $(\alpha < n)$ bzw.\\
    $|f(x)| \leq C ||x||^{-\alpha}$ fast überall in $\mathbb{R}^n \setminus B_{\epsilon}(0)$ mit $\alpha > n$\\
    $\implies f$ ist auf $B_{\epsilon}(0)$ bzw. $\mathbb{R}^n \setminus B_{\epsilon}(0)$ integrierbar
  \end{example}

  